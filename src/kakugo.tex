\section{ニコ書を支える覚悟}

\subsection{ニコ書の異常な安定性}

ニコ書はチーム、プロダクトともに非常に成功していると言えます。

具体的な例を上げると、2年間でコードに起因するサービス停止時間が3時間未満という短さであったり、
毎週安定したリリーススケジュールを保っていることであったり、
休日や深夜に緊急の対応をするようなトラブルがほとんど一度もないことであったり、
D社の中では異常とも言えるくらいの安定性を保っています。

特筆すべきは、2年間の間にほとんどのメンバーが入れ替わったにもかかわらず、
この安定性が全く低下していないことです。

この安定性はどのように作られているのでしょうか?

\subsection{技術を正しく運用するための覚悟}

もちろん安定したプロダクトを作るために、様々なテクノロジーを使っています。
テスト駆動開発であったり、アジャイルだったり、
モダンな開発環境には必須なことはひと通り導入しています。

しかし、それ以上に高度なテクノロジーは使っていません。
有償の開発支援ツールなどは一切使っていないし、
社内ルールに縛られて導入できないツールもたくさんあります。
きっと、世の中のモダンな開発をしている会社の環境とは、大差ないか少し悪い程度の違いしか無いでしょう。

ニコ書チームが他より圧倒的に優れている点はたった一つ。
それは覚悟の差だと思います。

\subsection{覚悟とはッ}

覚悟とは、暗闇の荒野に進むべき道を切り開くことです。
我々はプロジェクトという暗闇をテスト駆動開発で切り開いていきます。
テスト駆動開発正しく遂行するためには強い意志が必要です。

重要だと頭で理解していても、面倒だったり書きにくかったり、
ついついテストをサボってしまいたい時があるでしょう。
リリース予定が迫ってきて、偉い人から締め切りのプレッシャーを強く掛けられることもあるでしょう。
一部の環境だけテストが通らないけど、どう見ても正しく動いているからいいじゃないかと思うこともあるでしょう。

ニコ書チームはテストに関してだけは全く妥協を許しませんでした。
テストのないコードがマージされることはありません。
masterにマージされたコードが環境依存などでテストが落ちた場合、
全員が自分の作業を止めて問題の修正に当たります。
jenkinsとメンバー全員のマシンで1つでもテストが通らない場合はリリースを延期します。
たとえ偉い人からのプレッシャーがあったとしても、クビを覚悟で跳ね除けます。

自分を信じてはいけません。 人間は間違いを起こすのです。
テストがきちんと通って初めて、自分のコードに自信が持てるのです。
自信の無い仕事をユーザに提供することは絶対にできません。
自信を持ってコードをリリースするために、テストの覚悟が必須なのです。
そうして初めて、覚悟を「言葉」ではなく「心」で理解できるのです。

覚悟とは、暗闇の荒野に進むべき道を切り開くことです。
足元は無数の光に照らされ、影ひとつ無い道になっていないと転んでしまいます。
